\section{Motivation}
\label{sec:motivation}

In this section, we first develop some background on various TypeScript extensions and then present the main use case.

(TODO: ref https://www.typescriptlang.org/docs/handbook/advanced-types.html )

\subsection{String Literal Types}

String literal types allow you to attribute the exact value of a string at the type-level. For example, if we want the variable \ts{cat} to only hold the string \ts{"cat"}, then we can state \ts{let cat: "cat"}. Assigning any other string than \ts{"cat"} to \ts{cat}, for example \ts{cat = "dog"}, is a type error.

\subsection{Union Types}

Union types represent an untagged sum type. They are used to represent a type which can be either one type or another type, but there is no runtime tag which specifies which of the two it is. For example, if we have a variable which can either be the string \ts{"cat"} or \ts{"dog"}, then we can state that \ts{let catOrDog: "cat" | "dog"}. This uses the union type operator \ts{|} combined with the string literal types \ts{"cat"} and \ts{"dog"}. Both \ts{catOrDog = "cat"} and \ts{catOrDog = "dog"} are well-typed, but \ts{catOrDog = "bird"} is not.

\subsection{Flow-Sensitive If Statement}

A flow-sensitive if statement narrows the types in the scope of the then and else branch dependent on the tests in the statement. For example, if we start with \ts{catOrDog}, which has type \ts{"cat" | "dog"}, then we can test whether it is actually \ts{"cat"}. Within the then branch, the type of \ts{catOrDog} is narrowed to \ts{"cat"}, while in the else branch it is narrowed to \ts{"dog"}.

\begin{lstlisting}
let catOrDog: "cat" | "dog";
if (catOrDog === "cat") {
  // catOrDog is narrowed to type "cat" here
} else {
  // catOrDog is narrowed to type "dog" here
}
\end{lstlisting}

\subsection{Indexed Access Types}

An indexed access type is a specific form of type-level function. It allows to retrieve the type of a record type at a specific index, at the type-level. For example, if we have a record type \lstinline+type Cat = { lives: number, tag: "cat" }+, then we can query the type of the key \ts{lives} as \ts{type Lives = Cat["lives"]} which will be the type \ts{number}.

\subsection{Use Case}

The main use case we want to consider is that of creating a function for which the output type depends in the value of the input. For example, in the \ts{depFun} function below we have a function which takes a \ts{"t"} or \ts{"f"} string as input. The input value \ts{str} does not have type \ts{"t" | "f"} as might be expected, but instead has type \ts{T} which is a type variable bounded by \ts{"t" | "f"} (TODO: ref bounded quantification?). This is necessary since we want to refer to the type \ts{T} in the result type. This result type is \ts{F[T]} which encodes a type-level mapping for the output type dependent on the input type. In summary, the result type when called as \ts{depFun("t")} is \ts{number}, and \ts{boolean} when called as \ts{depFun("f")}.

\begin{lstlisting}
interface F {
  "t": number,
  "f": boolean,
}

function depFun<T extends "t" | "f">(str: T): F[T] {
  if (str === "t") {
    return 1;
  } else {
    return true;
  }
}

depFun("t"); // has type number
depFun("f"); // has type boolean
\end{lstlisting}

This style of pattern can be found in the official TypeScript bindings for certain browser APIs, for example the \ts{querySelector} function\footnote{\url{https://github.com/microsoft/TypeScript/blob/ca00b3248b1af2263d0223d68e792b7ca39abcab/lib/lib.dom.d.ts\#L11050}}. The relevant code which utilizes this pattern is copied below.

\begin{lstlisting}
interface HTMLElementTagNameMap {
  "a": HTMLAnchorElement;
  "abbr": HTMLElement;
  // ... and more
}

interface ParentNode {
  querySelector<K extends keyof HTMLElementTagNameMap>(
    selectors: K
  ): HTMLElementTagNameMap[K] | null;
}
\end{lstlisting}

The \ts{HTMLElementTagNameMap} is a record type containing the mapping of a HTML tag to the corresponding HTML element type resulting from the selection. The \ts{keyof} operator is another type-level operator which returns all keys of a record type as a union of string literals. So in this case it corresponds to \ts{"a" | "abbr" | ... } . With this pattern, callers of the \ts{querySelector} function are given more precise type information on what the function will return.
