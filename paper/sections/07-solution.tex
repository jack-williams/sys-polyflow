\section{Discussion of Solutions}
\label{sec:solution}

In the previous section, we saw that a naive solution to the problem does not work. To solve the problem there are two main paths forward: add additional syntax to ensure that the narrowing of the type parameter becomes safe, or, do a more constrained narrowing which does \emph{is} safe. We briefly discuss these in the next two sections

\subsection{Uniform Generics}

The first solution extends TypeScript with additional syntax for uniform generics. The programmer can write an exclamation mark \ts{!} on a generic type, which can only be instantiated with (TODO: check pr \url{https://github.com/microsoft/TypeScript/pull/30284} in more detail to give a better description) .

\begin{lstlisting}
// with Uniform Generics
function depFun<T! extends "t" | "f">(str: T): F[T] {
  if (str === "t") {
    return 1;
  } else {
    return true;
  }
}
\end{lstlisting}


\subsection{Safer Narrowing}

The second solution does a smarter narrowing within the branches to ensure that everything stays sound. Instead of replacing the type variable \ts{T} of a checked variable, we add a lower bound to the context. For instance in the \ts{depFun} example, in the \ts{str === "t"} branch we add the lower bound \ts{"t" extends T}. Using some additional typing rules, we are able to derive \ts{1 extends F[T]} and thus \ts{return 1} typechecks in this branch. This enables the wanted behaviour, while still disallowing the previous unsoundness.

\begin{lstlisting}
// with Safer Narrowing
function depFun<T extends "t" | "f">(str: T): F[T] {
  if (str === "t") {
    // type of str is narrowed to "t"
    // '"t" extends T' is added to the context
    // this allows '1 extends F[T]'
    return 1;
  } else {
    // type of str is narrowed to "f"
    // '"f" extends T' is added to the context
    // this allows 'true extends F[T]'
    return true;
  }
}
\end{lstlisting}

In the \ts{depFun2} example, the unsoundness is avoided. The added constraint \ts{"t" extends T} is still a valid claim if \ts{str2} is the value \ts{"f"}. Because in that case the type \ts{T} can really only be \ts{"t" | "f"}, and \ts{"t" extends "t" | "f"} is valid. By the same reasoning as in the previous example, we are able to do \ts{return 1}, but this behaviour is sound since the caller of the function will be expecting \ts{number} or \ts{number | boolean} depending on whether \ts{T} is \ts{"t"} or \ts{"t" | "f"}.

\begin{lstlisting}
function depFun2<T extends "t" | "f">(str: T, str2: T): F[T] {
  if (str === "t") {
    // type of str is narrowed to "t"
    // '"t" extends T' is added to the context
    // return 1 is safe, since caller will expect either number or number | boolean
    return 1;
  } else {
    // ...
  }
}

depFun("t", "t"); // T = "t" , return type is number
depFun("t", "f"); // T = "t" | "f" , return type is number | boolean
\end{lstlisting}

TODO: add note on difference between specification with lower bounds and actual implementation in compiler, or somewhere else?

\subsection{Comparison}

Pros/Cons of both approaches

Uniform Generics
\\- Pro: more general interaction with dependent-type-like features
\\- Con: additional syntax

Safer Narrowing
\\- Pro: no additional syntax
\\- Con: unsure on interaction with future features
